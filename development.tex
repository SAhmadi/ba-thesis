\chapter{Entwicklung}
Um das Serverless Konzept mit seinen Vor- und Nachteil
besser zu verdeutlichen, wird eine prototypische Anwendung nach den Serverless-Prinzipien
entwickelt. Dieses Kapitel beinhaltet die Architektur und Implementierung
der Anwendung. Davor werden Dienste und Werkzeuge vorgestellt,
die für die Realisierung verwendet wurden.

Es soll den Korrektoren und Studenten des Moduls
Programmierung der Heinrich-Heine-Universität
möglich sein einfache Java Programme in einer einheitlichen und isolierten Umgebung
auszuführen und zu testen.
Studenten können dadurch ihre Aufgaben vor der Abgabe testen, um sicherzugehen,
dass sie auch bei der Korrektur wie gewollt laufen.
Korrektoren müssen sich keine Gedanken um die Sicherheit ihrer Systeme
machen. Sie brauchen keine aufwändigen VMs starten, denn die Aufgaben werden
bereits in einer abgesicherten Umgebung ausgeführt.

Die serverlose Anwendung ermöglicht es einzelne Java Dateien auszuführen.
Werden zusätzliche JUnit Testdateien mitgeliefert,
können auch diese gestartet werden. Das Resultat der Tests wird wieder
zurückgeliefert.
Da die Aufgaben des Moduls Gradle benutzen,
können auch Gradle-Projekte ausgeführt und getestet werden.
Jedoch wird dafür das Applikation-Plugin\footnote{\url{https://docs.gradle.org/current/userguide/application_plugin.html}}
vorausgesetzt, sodass die Aufgaben des Moduls leicht angepasst werden müssen.
Die Anwendung kann vollständig durch die Kommandozeile angesprochen werden, denn
sie stellt eine REST-API bereit.
Jedoch wird auch ein Endpunkt zur Verfügung gestellt, der eine Webseite zurückliefert.
Das Layout der Seite richtet sich grob nach dem Styleguide der
Modularen Online Platform für Studierende (MOPS)\footnote{\url{https://mops.style}}
der HHU. In ihr wird das Frontend eingebunden.
Es kann von den Studenten als Entwicklungsumgebung benutzt werden, um
ihre Java-Projekte zu entwickeln und anschließend zu testen.
Jedoch liegt der Fokus der Arbeit auf der serverlosen Backend-Anwendung.


\section{Dienste und Tools}
\paragraph{Serverloser Dienst} Bei dem serverlosen Dienst wird auf Google Cloud Run zurückgegriffen.
Es bietet mehr Freiheit als FaaS Dienste der Cloud-Provider an, da mittels Dockerfile die Laufzeitumgebung
der Anwendung genaustens definiert und konfiguriert werden kann. Auch Open-Source
Plattformen wurden in Erwägung gezogen, jedoch aufgrund der etwas komplexeren Bereitstellung nicht ausgewählt.
Da die Anwendung für Studenten und Korrektoren
des Moduls Programmierung ausgelegt ist und der Datenschutz für Universitäten eine wichtige Rolle
spielt, wird der Punkt in \autoref{chap:conclusion_and_outlook} noch einmal aufgegriffen.

\paragraph{Webserver} Die Anwendung selbst wird mit dem Web-Framework
Flask\footnote{\url{https://palletsprojects.com/p/flask/}}
entwickelt. Als Programmiersprache wird Python verwendet.
Flask ermöglicht die schnelle und unkomplizierte Bereitstellung einer REST-API.
Da MOPS auf Spring Boot\footnote{\url{https://spring.io/projects/spring-boot}} basiert und
der Styleguide die Template-Engine Thymeleaf verwendet, welche in Flask nicht unterstützt wird,
richtet sich die Anwendung nur an dem Styleguide und implementiert es nicht vollständig.

\paragraph{Frontend Web-Framework} Für die grafische Entwicklungsumgebung
wird mit Vue.js\footnote{\url{https://vuejs.org}} ebenfalls ein Web-Framework verwendet.
Das Frontend wird in der Programmiersprache JavaScript geschrieben. Es dient als
erweiterte Funktionalität und gehört nicht zum Kern der in dieser Arbeit entwickelten
serverlosen Anwendung. Jedoch bietet sie eine bequemere Interaktion mit der Anwendung als
die Kommandozeile. Das Frontend wird mit
Firebase Hosting\footnote{\url{https://firebase.google.com/docs/hosting}}, einem weiteren Google Dienst,
bereitgestellt. Bei Firebase Hosting kann das sichere Hosten von statischen Webseiten
mit wenigen Befehlen bewältigt werden.

\section{Architektur}
Die Architektur in Abbildung~\ref{fig:architecture} besteht auf zwei
Komponenten.
Die Cloud Run Instanzen bilden das serverlose Backend.
Eine Instanz stellt Endpunkte bereit die mit HTTP-Anfragen\footnote{\url{https://tools.ietf.org/html/rfc7540}}
angesprochen werden
können. Rufen Benutzer mittels einer HTTP-GET Anfrage den
Root-Endpunkt auf \textbf{(1)} so wird
die \texttt{index.html} Datei zurückgesendet \textbf{(2)}.
Sie enthält die grafische Benutzeroberfläche. Das Layout richtet
sich an dem Styleguide des MOPS. Durch ein
HTML-iframe Element\footnote{\url{https://developer.mozilla.org/en-US/docs/Web/HTML/Element/iframe}}
wird das eigentliche Frontend eingebunden. Das Frontend dient als Entwicklungsumgebung für
Studenten, um ihre Programme zu entwickeln. Die Aufteilung von Frontend und Backend
ermöglicht eine einfache Verwaltung und Erweiterung beider Komponenten.
Durch Interaktion mit der Entwicklungsumgebung \textbf{(3)} lassen sich einige
der restlichen Endpunkte \textbf{(4)} aufrufen. 
Da das Backend eine REST-API bereitstellt, ist auch ein direkter Zugriff \textbf{(5)} auf alle 
Endpunkte über die Kommandozeile und HTTP-POST Anfragen möglich.

\begin{figure}
  \centering
  \includegraphics[height=10cm]{fig/architecture.pdf}
  \caption{Architektur und Interaktion der Anwendung}
  \label{fig:architecture}
\end{figure}

\section{Serverless Backend}
Das Backend wird durch Cloud Run, dem Serverless Container Dienst von Google, zur Verfügung gestellt.
Der Dienst bietet bei der Anwendungsentwicklung viel Freiheit, denn 
neben dem eigentlichen Quellcode müssen in einer Dockerfile alle Abhängigkeiten der Anwendung
definiert werden. Dies reicht für die Bereitstellung bereits aus.

Für eine einfache Interaktion mit dem Dienst wird das
\emph{Google Cloud SDK}\footnote{\url{https://cloud.google.com/sdk/docs?hl=de}} verwendet.
Es stellt mit \texttt{gcloud} eine Kommandozeilen-Anwendung bereit.
Mit ihr können viele der Google Cloud Dienste über die Kommandozeile angesprochen werden,
was die Bereitstellung der Anwendung vereinfacht.
Bevor die Anwendung auf Cloud Run bereitgestellt werden kann, muss in der
Google Cloud Konsole ein Projekt erstellt und eingerichtet werden \cite{CloudRunInstructions}.\\

\begin{lstlisting}[caption={Google Cloud SDK einrichten}, label={lst:gcloud_init}]
$ gcloud init
$ gcloud config set project [PROJECT_ID]
$ gcloud services enable containerregistry.googleapis.com \            
    cloudbuild.googleapis.com \
    run.googleapis.com
\end{lstlisting}

Danach wird \texttt{gcloud} selbst initialisiert und autorisiert \textbf{(Zeile 1)}.
Außerdem muss das zuvor erstellte Projekt als das aktuelle Projekt gekennzeichnet werden \textbf{(Zeile 2)}.
Abschließend werden die für die serverlose Anwendung  benötigten Dienste aktiviert \textbf{(Zeile 3)}.
Dabei handelt es sich neben dem eigentlichen Serverless Container Dienst um
Google Container Registry\footnote{\url{https://cloud.google.com/container-registry?hl=de}}
und Google Cloud Build\footnote{\url{https://cloud.google.com/cloud-build?hl=de}}.
Cloud Build erzeugt aus der Dockerfile unter anderem das Dockerimage, während Container Registry
es anschließend für Cloud Run bereitstellt.

\subsection{Dockerfile}
\paragraph{Java Development Kit} Für das Kompilieren und anschließende Ausführen der Java Dateien wird
Zulu OpenJDK 11 von Azul benutzt, da die Installation auf
Linux-Betriebssystem gut dokumentiert wurde \cite{AzulZuluJDK}.
Das JDK stellt die zwei Befehle \texttt{javac} und \texttt{java} bereit.
Übergeben Student oder Korrektoren ihren Java Quellcode, so muss das Backend in der Lage
sein, den Befehl \texttt{javac} aufzurufen. Denn dadurch können die Java Dateien kompiliert werden.
Nach der erfolgreichen Kompilation können durch den \texttt{java} Befehl die kompilierten Dateien
ausgeführt werden.
Um dem Backend dies zu Ermöglichen muss das JDK in die Laufzeitumgebung eingepackt werden,
damit es während der Ausführung die zwei Befehle aufrufen kann.\\

\begin{lstlisting}[caption={Ausschnitt aus der Dockerfile mit JDK}, label={lst:dockerfile}]
# ... 

# Get Azul's OpenJDK
RUN wget https://cdn.azul.com/public_keys/alpine-signing@azul.com-5d5dc44c.rsa.pub

# ...

# Install Zulu11
RUN apk --no-cache add zulu11-jdk-headless
ENV JAVA_HOME=/usr/lib/jvm/default-jvm
ENV PATH="$JAVA_HOME/bin:${PATH}"

# Check if two commands are available
RUN which javac
RUN which java

# ...
\end{lstlisting}

In Auflistung~\ref{lst:dockerfile} ist ein Ausschnitt
aus der Dockerfile zusehen, welche die Laufzeitumgebung
der Cloud Run Instanzen definiert. Zuerst wird das JDK
heruntergeladen \textbf{(Zeile 4)}. Dann wird es
installiert \textbf{(Zeile 9)} und wichtige Umgebungsvariablen
werden gesetzt \textbf{(Zeile 10, 11)}.
Abschließend wird geprüft ob die beiden Befehle
zur Verfügung stehen \textbf{(Zeile 14, 15)}.

\paragraph{Gradle} Um ein Gradle-Projekte auszuführen und zu testen,
gibt es in der unter anderem zwei Möglichkeiten. Einem Gradle-Projekt
kann der Gradle-Wrapper (Wrapper)\footnote{\url{https://docs.gradle.org/current/userguide/gradle_wrapper.html}}
beigelegt sein. Dieser definiert welche Version von Gradle benutzt
werden soll und läd sich die benötigten Komponenten für das Kompilieren
und Bauen des Projekts selbstständig herunter \cite{GradleWrapper}.
Normalerweise ist es empfohlen den Gradle-Wrapper zu benutzen,
da dieser das Bauen und Kompilieren des Projektes standardisiert und
somit weniger Fehler entstehen können \cite{GradleWrapper}.
Jedoch ist der Wrapper für das serverlose Backend etwas ungünstig.
Bei einer Anfrage, in der eine neue Cloud Run Container Instanz
gestartet wird, läd sich der Wrapper erst einmal die benötigten
Gradle-Komponenten herunter. Dadurch braucht die Bearbeitung der
Anfrage



\subsection{Endpunkte}
At vero eos et accusam et justo duo dolores et ea rebum. Stet clita kasd gubergren,
no sea takimata sanctus est Lorem ipsum dolor sit amet. Lorem ipsum dolor sit amet,
consetetur sadipscing elitr, sed diam nonumy eirmod tempor invidunt ut labore et
dolore magna aliquyam erat, sed diam voluptua.
At vero eos et accusam et justo duo dolores et ea rebum.
Stet clita kasd gubergren, no sea takimata sanctus est Lorem ipsum dolor sit amet.

\section{Frontend}
At vero eos et accusam et justo duo dolores et ea rebum. Stet clita kasd gubergren,
no sea takimata sanctus est Lorem ipsum dolor sit amet. Lorem ipsum dolor sit amet,
consetetur sadipscing elitr, sed diam nonumy eirmod tempor invidunt ut labore et
dolore magna aliquyam erat, sed diam voluptua.
At vero eos et accusam et justo duo dolores et ea rebum.
Stet clita kasd gubergren, no sea takimata sanctus est Lorem ipsum dolor sit amet.
