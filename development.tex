\chapter{Entwicklung}
Um das Serverless Konzept mit seinen Vor- und Nachteil
besser zu verdeutlichen, wird eine prototypische Anwendung nach den Serverless-Prinzipien
entwickelt. Dieses Kapitel beinhaltet die Architektur und Implementierung
der Anwendung. Davor werden Dienste und Werkzeuge vorgestellt,
die für die Realisierung verwendet wurden.

Es soll Benutzern der Anwendung möglich sein einfache, Kommandozeile-basierte
Java Programme in einer einheitlichen und isolierten Umgebung
auszuführen und zu testen. Der Dienst kann auch von Studenten und
Korrektoren benutzt werden.
Studenten können dadurch ihre Aufgaben vor der Abgabe testen, um sicherzugehen,
dass sie auch bei der Korrektur wie gewollt laufen.
Korrektoren müssen sich keine Gedanken um die Sicherheit ihrer Systeme
machen. Sie brauchen keine aufwändigen VMs starten, denn die Aufgaben werden
bereits in einer abgesicherten Umgebung ausgeführt.

Die serverlose Anwendung ermöglicht es einzelne Java Dateien auszuführen.
Werden zusätzliche JUnit5\footnote{\url{https://junit.org/junit5/docs/current/user-guide/}}
Testdateien mitgeliefert, können auch die Tests gestartet werden.
Das Resultat der Tests wird wieder an den Benutzer zurückgeliefert.
Da die Aufgaben des Moduls Gradle benutzen,
können auch Gradle-Projekte ausgeführt und getestet werden.
Jedoch verwenden die Modulaufgaben keine JUnit Tests, sodass sie angepasst werden müssen.
Außerdem wird für das Ausführen der Gradle-Projekte
das Applikations-Plugin\footnote{\url{https://docs.gradle.org/current/userguide/application_plugin.html}}
vorausgesetzt. Die Anwendung kann vollständig durch die Kommandozeile angesprochen werden, denn
sie stellt eine REST-API bereit.
Jedoch wird auch ein Endpunkt zur Verfügung gestellt, der eine Webseite zurückliefert.
Das Layout der Seite richtet sich grob nach dem Styleguide der
Modularen Online Platform für Studierende (MOPS)\footnote{\url{https://mops.style}}
der HHU. In ihr wird das Frontend eingebunden.
Es kann von den Studenten als Entwicklungsumgebung benutzt werden, um
ihre Java-Projekte zu entwickeln und anschließend zu testen.
Jedoch liegt der Fokus der Arbeit auf der serverlosen Backend-Anwendung.


\section{Dienste und Tools}
\paragraph{Serverloser Dienst} Bei dem serverlosen Dienst wird auf Google Cloud Run zurückgegriffen.
Es bietet mehr Freiheit als FaaS Dienste der Cloud-Provider an, da mittels Dockerfile die Laufzeitumgebung
der Anwendung genaustens definiert und konfiguriert werden kann. Auch Open-Source
Plattformen wurden in Erwägung gezogen, jedoch aufgrund der etwas komplexeren Bereitstellung nicht ausgewählt.
Da die Anwendung für Studenten und Korrektoren
des Moduls Programmierung ausgelegt ist und der Datenschutz für Universitäten eine wichtige Rolle
spielt, wird der Punkt in \autoref{chap:conclusion_and_outlook} noch einmal aufgegriffen.

\paragraph{Webserver} Die Anwendung selbst wird mit dem Web-Framework
Flask\footnote{\url{https://palletsprojects.com/p/flask/}}
entwickelt. Als Programmiersprache wird Python verwendet.
Flask ermöglicht die schnelle und unkomplizierte Bereitstellung einer REST-API.
Da MOPS auf Spring Boot\footnote{\url{https://spring.io/projects/spring-boot}} basiert und
der Styleguide die Template-Engine Thymeleaf verwendet, welche in Flask nicht unterstützt wird,
richtet sich die Anwendung nur an dem Styleguide und implementiert es nicht vollständig.

\paragraph{Frontend Web-Framework} Für die grafische Entwicklungsumgebung
wird mit Vue.js\footnote{\url{https://vuejs.org}} ebenfalls ein Web-Framework verwendet.
Das Frontend wird in der Programmiersprache JavaScript geschrieben. Es dient als
erweiterte Funktionalität und gehört nicht zum Kern der in dieser Arbeit entwickelten
serverlosen Anwendung. Jedoch bietet sie eine bequemere Interaktion mit der Anwendung als
die Kommandozeile. Das Frontend wird mit
Firebase Hosting, einem weiteren Google Dienst,
bereitgestellt. Bei Firebase Hosting kann das sichere Hosten von statischen Webseiten
mit wenigen Befehlen bewältigt werden \cite{FirebaseHosting}.

\section{Architektur}
Die Architektur in Abbildung~\ref{fig:architecture} besteht auf zwei
Komponenten.
Die Cloud Run Container Instanzen bilden das serverlose Backend.
Es werden Endpunkte bereitgestellt, die mit
HTTP-Anfragen\footnote{\url{https://tools.ietf.org/html/rfc2616}}
angesprochen werden können. Rufen Benutzer
mittels einer HTTP-GET Anfrage den
Root-Endpunkt auf \textbf{(1)} so wird
die \texttt{index.html} Datei zurückgesendet \textbf{(2)}.
Sie enthält die grafische Benutzeroberfläche. Das Layout richtet
sich an dem Styleguide des MOPS. Durch ein
HTML-iframe Element\footnote{\url{https://developer.mozilla.org/en-US/docs/Web/HTML/Element/iframe}}
wird das eigentliche Frontend eingebunden. Das Frontend dient als Entwicklungsumgebung für
Studenten, um ihre Programme zu entwickeln. Die Aufteilung von Frontend und Backend
ermöglicht eine einfache Verwaltung und Erweiterung beider Komponenten.
Durch Interaktion mit der Entwicklungsumgebung \textbf{(3)} lassen sich einige
der restlichen Endpunkte \textbf{(4)} aufrufen. 
Da das Backend eine REST-API bereitstellt, ist auch ein direkter Zugriff \textbf{(5)} auf alle 
Endpunkte über die Kommandozeile durch HTTP-POST Anfragen möglich.

\begin{figure}
  \centering
  \includegraphics[height=10cm]{fig/architecture.pdf}
  \caption{Architektur und Interaktion der Anwendung}
  \label{fig:architecture}
\end{figure}

\section{Serverless Backend}
Das Backend wird durch Endpunkte angesprochen.
Der Java Quellcode kann mittels HTTP-POST Anfrage an das
Backend gesendet werden und wird dort durch das Java Development Kit
beziehungsweise Gradle ausgeführt oder getestet.


\paragraph{Java Development Kit} Für das Kompilieren und
anschließende Ausführen der Java Dateien wird
Zulu OpenJDK 11 benutzt, da die Installation auf
Alpine Linux-Betriebssystemen\footnote{\url{https://alpinelinux.org}}
verständlich dokumentiert ist \cite{AzulZuluJDK}.
Das JDK stellt die zwei Befehle \texttt{javac} und \texttt{java} bereit.
Übergeben Studenten oder Korrektoren ihren Java Quellcode mittels HTTP-POST Anfrage an
das Backend, so muss es in der Lage sein, den Befehl \texttt{javac}
aufzurufen. Denn dadurch können die Java Dateien kompiliert werden.
Nach der erfolgreichen Kompilation können durch den \texttt{java} Befehl die kompilierten Dateien
ausgeführt werden. Um dem Backend dies zu Ermöglichen muss das JDK in die Laufzeitumgebung eingepackt
werden, damit die Anwendung während ihrer Ausführung die zwei Befehle aufrufen kann.\\

\begin{lstlisting}[caption={Ausschnitt aus der Dockerfile. Herunterladen und Einrichten des JDK.}, label={lst:dockerfile}, escapechar=|]
# Install Zulu11
RUN apk --no-cache add zulu11-jdk-headless |\label{line:installJDK}|
ENV JAVA_HOME=/usr/lib/jvm/default-jvm |\label{line:javaHome}|
ENV PATH="$JAVA_HOME/bin:${PATH}" |\label{line:javaPath}|

# Check if two commands are available
RUN which javac |\label{line:testJavac}|
RUN which java |\label{line:testJava}|
\end{lstlisting}

In Auflistung~\ref{lst:dockerfile} ist ein Ausschnitt
aus der Dockerfile zusehen, welche die Laufzeitumgebung
der Cloud Run Container Instanzen definiert. Das JDK wird
heruntergeladen und installiert \textbf{(Zeile~\ref{line:installJDK})}.
Außerdem werden wichtige Umgebungsvariablen
gesetzt \textbf{(Zeile~\ref{line:javaHome},~\ref{line:javaPath})}.
Abschließend wird geprüft ob die beiden Befehle
zur Verfügung stehen \textbf{(Zeile~\ref{line:testJavac},~\ref{line:testJava})}.

\paragraph{JUnit Console Launcher} Da der Endpunkt \texttt{/test/java} einzelne Javadateien
und keine vollständigen Gradle-Projekte entgegennimmt, müssen die mitgelieferten Tests
manuell gestartet werden. Es wird der JUnit Console Launcher benutzt. Mit diesem
Kommandozeilen-Programm können JUnit Tests ausgeführt werden.
Hierfür wird das eigenständige Console Launcher
Jar-Archiv\footnote{\url{https://de.wikipedia.org/wiki/Java_Archive}} beim Ausführen
des \texttt{java} Befehls mit eingebunden \cite{JunitConsoleLauncher}.
Um dies zu ermöglichen, muss es der Anwendung zur Laufzeit bereitstehen.
Deshalb wird auch das Jar-Archiv in die Laufzeitumgebung verpackt.

\paragraph{Gradle} Damit Gradle-Projekte gestartet und getestet
können wird, wie in Auflistung~\ref{lst:gradle} zu sehen, Gradle in die
Laufzeitumgebung eingepackt. Es reicht aus Gradle herunterzuladen
\textbf{(Zeile~\ref{line:downloadGradle})}
und zu entpacken \textbf{(Zeile~\ref{line:unzipGradle})}. Somit
hat die Anwendung während der Laufzeit Zugriff auf den \texttt{gradle} Befehl.
Die Anwendung braucht nur noch den \texttt{gradle test} oder \texttt{gradle run} Befehl
aufrufen und das Gradle-Projekt kann getestet oder ausgeführt werden \cite{GradleCLI}.\\

\begin{lstlisting}[caption={Ausschnitt aus der Dockerfile. Herunterladen und Entpacken von Gradle.}, label={lst:gradle}, escapechar=|]
# Get Gradle distribution zip
RUN wget https://downloads.gradle-dn.com/distributions/gradle-6.3-bin.zip |\label{line:downloadGradle}|
RUN unzip gradle-6.3-bin.zip |\label{line:unzipGradle}|
\end{lstlisting}

\subsection{Endpunkte}
Die Endpunkte sind mit \texttt{/run} und
\texttt{/test} in zwei Gruppen eingeteilt.
Die erste Gruppe ist für das Kompilieren und anschließende
Ausführen von Java Quellcode zuständig. Die zweite Gruppe ist
für das Ausführen von JUnit-Tests verantwortlich. Die Anwendung
sendet das Result der Ausführung als eine HTML-Datei zurück. Es kann
jedoch mit dem zusätzlichen Query-String\footnote{\url{https://wiki.selfhtml.org/wiki/Query_String}}
\texttt{?return=json} auch ein JSON-Objekt zurückgesendet werden.

\paragraph{Ausführen und Testen einzelner Javadateien} Ein beispielhafter Aufruf des
\texttt{/run/java} Endpunktes mit curl\footnote{\url{https://curl.haxx.se}} ist in
Auflistung~\ref{lst:curlRunJava} dargestellt.
Die einzelnen Java Dateien werden im
Entity-Body\footnote{\url{https://tools.ietf.org/html/rfc2616\#section-7.2}}
der Anfrage mitgeliefert
\textbf{(Zeile~\ref{line:javaFile1},~\ref{line:javaFile2})}.
Zusätzlich muss der Name der Klasse
angegeben werden, in der die Hauptmethode definiert ist
\textbf{(Zeile~\ref{line:mainClass})}.
Dieser Endpunkt soll den Studenten ermöglichen ihren Quellcode
schnell und unkompliziert auszuführen.\\

\begin{lstlisting}[
  caption={HTTP-POST Anfrage an \texttt{/run/java} um einzelne Java Dateien zu kompilieren und auszuführen.},
  label={lst:curlRunJava},
  escapechar=|
]
curl -F main_file=MyMainClass \ |\label{line:mainClass}|
  -F file=@MyMainClass.java \ |\label{line:javaFile1}|
  -F file=@Calculator.java \ |\label{line:javaFile2}|
  https://ba-serverless-testing-t6e6p4w6oa-ew.a.run.app/run/java
\end{lstlisting}

Für das Ausführen von JUnit-Tests müssen die Testdateien, wie in Auflistung~\ref{lst:curlTestJava},
mitgeliefert werden \textbf{(Zeile~\ref{line:testFile})}.
Zusätzlich muss der Endpunkt zu \texttt{/test/java} umgeändert werden
\textbf{(Zeile~\ref{line:testJavaUrl})}.\\

\begin{lstlisting}[
  caption={HTTP-POST Anfrage an \texttt{/test/java} um einzelne JUnit-Tests auszuführen.},
  label={lst:curlTestJava},
  escapechar=|
]
curl -F file=@Calculator.java|\label{line:javaFile}| -F file=@CalculatorTest.java \ |\label{line:testFile}|
  https://ba-serverless-testing-t6e6p4w6oa-ew.a.run.app/test/java |\label{line:testJavaUrl}|
\end{lstlisting}

\paragraph{Ausführen und Testen eines Gradle-Projekts} Um ein Gradle-Projekt
über dem Endpunkt \texttt{run/gradle} auszuführen,
muss das Projekt vorher in ein Zip-Archiv komprimiert werden.
Erst dann kann es der HTTP-POST Anfrage beigelegt \textbf{(Zeile~\ref{line:gradleZip})}
und an das Backend versendet werden.\\

\begin{lstlisting}[
  caption={HTTP-POST Anfrage an \texttt{/run/gradle} um ein Gradle-Projekt auszuführen.},
  label={lst:curlRunGradle},
  escapechar=|
]
curl -F file=@gradle_project.zip \ |\label{line:gradleZip}|
  https://ba-serverless-testing-t6e6p4w6oa-ew.a.run.app/run/gradle
\end{lstlisting}

Soll das Projekt getestet werden, sodass enthaltene JUnit-Tests ausgeführt werden,
muss nur der Endpunkt des Befehls zu \texttt{/test/gradle} abgeändert werden.
Diese beiden Endpunkte können für das Ausführen der Aufgaben im Modul Programmierung
der HHU verwendet werden.

\paragraph{Einbinden des Frontends} Es gibt noch einen weiteren Endpunk, welcher
nicht primär für das Testen und Ausführen von Java-Quellcode zuständig ist.
Durch eine GET-Anfrage an den \texttt{/} Endpunkt wird eine grafische Benutzeroberfläche zurückgesendet.
Vor Beginn der Ausarbeitung war als eine mögliche Erweiterung definiert, dass die serverlose Anwendung an
das MOPS System der HHU angebunden werden soll. Da das Einbinden an das System nicht
das zentrale Thema der Arbeit ist, wurde nur das Layout nachgebaut.
Jedoch wurde zusätzlich eine Entwicklungsumgebung gebaut.
Die Entwicklungsumgebung ist das eigentliche Frontend und wird durch einen HTML-iframe
in das nachgebaute MOPS Layout eingebunden. Somit kann sie auch vollkommen eigenständig benutzt werden.

\subsection{Ablauf}
\begin{figure}
  \centering
  \includegraphics[height=8cm]{fig/ablauf_run_java.pdf}
  \caption{Ablauf einer Anfrage an \texttt{/run/java} Endpunkt.}
  \label{fig:ablauf_run_java}
\end{figure}

In Abbildung~\ref{fig:ablauf_run_java} ist der Ablauf einer Anfrage an den Endpunkt
\texttt{/run/java} abgebildet. Nachdem das Backend die Anfrage entgegennimmt,
werden die enthaltenen Dateien in einem temporären Bereich gesichert.
Von dort aus wird durch das Subprocess-Modul von Python der \texttt{javac} Befehl aufgerufen.
Mit dem Subprocess-Modul lassen sich zusätzliche Prozesse startet und verwalten \cite{PythonSubprocess}.
Wenn das Kompilieren des Quellcodes erfolgreich ist, wird mit dem \texttt{java} Befehl
das Programm gestartet. Die Standardausgabe des Programms wird eingefangen und
dem Benutzer als Antwort zurückgesendet. Für die beiden Gradle Endpunkte können
ist der Ablauf ähnlich. Es können die beiden JDK Befehle durch den \texttt{gradle}
Befehl ausgetauscht werden. Soll das übermittelte Gradle-Projekt gebaut und ausgeführt
werden, wird der \texttt{gradle run} Befehl durch das Subprocess-Modul aufgerufen.
Soll es gebaut und getestet werden, wird der \texttt{gradle test} Befehl aufgerufen.

\section{Frontend}
Für das Erstellen der Entwicklungsumgebung wurde das Open Source Framework \emph{Vue.js}
und der Texteditor \emph{Monaco Editor} verwendet. Monaco Editor wird unter anderem
bei VS Code\footnote{\url{https://code.visualstudio.com}} verwendet. Es stellt einige übliche Funktionen,
wie Syntax-Highlighting \cite{MonacoEditor} bereit.
Jedoch wurde für eine angenehmere Benutzererfahrung eine zusätzliche Projektansicht
entwickelt, sodass Benutzer zwischen unterschiedlichen Dateien wechseln können.
Das Frontend setzt sich aus vier Komponenten zusammen. Die erste Komponente beinhaltet
die Buttons für das Senden des Gradle-Projekts an das Backend. Die zweite Komponente
ist die Projektansicht. Das Projekt wird als Baumstruktur dargestellt. Außerdem können
Dateien und Ordner erstellt werden. Die dritte Komponente ist der Editor. Hier
können Benutzer ihren Quellcode schreiben. Nach dem Ausführen oder Testen wird die Antwort
des Backends in der Konsole, der vierten Komponente, ausgegeben.
Das Frontend stellt den Benutzern das Grundgerüst eines Gradle-Projekts bereit.
Benutzer können durch das Erstellen von neuen Dateien und Ordnern, ihre
Anwendungslogik in das Projekt einfügen und das Gerüst ausbauen. Das Frontend bietet
die nötigsten Funktionen, um Benutzern zu ermöglichen ihren Quellcode
zu starten und zu testen. Es ist ein erster Ansatz einer Entwicklungsumgebung
und ist noch ausbaufähig.

Das Frontend wird durch Google Firebase Hosting, zur Verfügung gestellt.
Mit wenigen Befehlen ermöglicht der Dienst das unkomplizierte Bereitstellen von statische Webseiten.
Unter anderem ist SSL\footnote{\url{https://tools.ietf.org/html/rfc6101}}
vorkonfiguriert, sodass beim Besuchen der Webseite eine sichere
Verbindung aufgebaut wird \cite{FirebaseHosting}.


