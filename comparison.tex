\chapter{Ähnliche Anwendungen}

\section{Repl.it}
Repl.it ist eine Entwicklungsumgebung im Browser \cite{Replit}. Es können
entweder vordefinierte Umgebungen gestartet oder ein Projekt aus
einer GitHub Repository importiert werden. Bei den Umgebungen
können unterschiedlichen Programmiersprachen und Frameworks ausgewählt
werden. Die Projektstruktur und der Quellcode werden dann in einer grapfischen Oberfläche
angezeigt. Zusätzlich wird eine Konsole bereitgestellt, wodurch die
verbundene Maschine in der Cloud kontrolliert werden kann. In der kostenlosen
Version werden 500 Megabyte an Arbeitsspeicher und Festplattenspeicher,
sowie 0,2 bis 0,5 vCPUs zur Verfügung gestellt.
Anders als die ausgearbeitete Anwendung wird also eine konstante Verbindung
zu einer VM in der Cloud bereitgestellt.
Repl.it hat den großen Vorteil, dass der Quellcode bereits auf dem System liegt,
wodurch die Ausführung schneller läuft als bei dem serverlosen Dienst.
Bei der ausgearbeiteten serverlosen Anwendung müssen die Programme
immer wieder zu einem Zip-Archiv komprimiert und versendet werden.
Repl.it bietet noch weitere Funktionen, wie kollaboratives Arbeiten \cite{Replit}.
Es ist eine sehr gute und günstige Alternative und kann als Vorlage
für Verbesserungen der selbst entwickelten Anwendung dienen.

\section{GitHub Codespaces}
Ein recht neuer Dienst ist GitHub Codespaces.
Es handelt sich auch hier um eine Entwicklungsumgebung in der Cloud \cite{GithunCodespaces}.
Es können GitHub Repositories importiert und im Browser oder
in VS Code geöffnet werden. Die Entwicklungsumgebung
basiert auf VS Code und bietet dadurch zahlreiche Funktionen \cite{GithunCodespacesDocs}.
Der Dienst ist zum Zeitpunkt dieser Arbeit nur in begrenzter Public Beta, sodass
er nicht ausprobiert und weiter verglichen werden kann.