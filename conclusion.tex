\chapter{Fazit}

Im Rahmen dieser Arbeit wurde eine serverlose Anwendung entwickelt,
die es Benutzern ermöglicht einfache Java Konsolenanwendungen in
einer sicheren Umgebung auszuführen und zu testen. Sie
lässt sich primär durch eine REST-API steuern, kann aber auch
zum Teil über die grafische Entwicklungsumgebung angesprochen
werden. Um einen Überblick über das relativ neue Konzept des
Serverless-Computing zu geben, wurden die wichtigsten Eigenschaften,
sowie Vor- und Nachteile erläutert.

Das Serverless Prinzip ist für die Anwendung nicht ganz
geeignet. Während der Evaluation der Anwendung wurden einige Probleme
ersichtlich. Die Anwendung weist bei den Gradle Endpunkten noch
einen Fehler auf, sodass bei großer Nachfrage viele Anfragen
nicht erfolgreich beantwortet werden. Der Vorteil der
einfachen Skalierung von Serverless wird hier nicht ausgenutzt und
muss noch verbessert werden. Die Antwortzeit der Anwendung ist noch zu groß,
um eine angenehme Benutzererfahrung zu bieten. Das
auszuführende Programm muss bei jeglichen Änderungen
komprimiert und immer wieder an den Dienst versendet
werden. Außerdem sind serverlose Anwendungen zustandslos.
Eine Interaktion mit den ausgeführten Programmen wird nur durch
Umstände möglich. Anders ist dies bei der konstanten Verbindung zu
einer VM in der Cloud. Der Quellcode kann auf der Maschine
liegen und muss deshalb nicht immer wieder versendet werden.
Durch die direkte Verbindung können auch mit den ausgeführten Programmen
interagiert werden. Der Sicherheitsaspekt der Arbeit ist auch mit einer VM basierten
Cloud Lösung gegeben, ähnlich wie Repl.it oder GitHub Codespaces.
Die Entwicklung der Applikation hat deutlich gemacht,
dass die Serverless Prinzipien nicht für alle Anwendungen vorteilhaft sind.

Der Dienst hat aber nicht nur Nachteile.
Bei serverlose Anwendungen wird nur der eigentliche Verbrauch
in Zahlung gestellt. Da die entwickelte Anwendung auch
auf null aktive Instanzen herunterskalieren kann, fallen bei nicht Benutzen
der Anwendung keine kosten an. Hier hat die serverlose Anwendung einen
Vorteil gegenüber VM basierten Lösungen, wie Repl.it oder
GitHub Codespaces. Nach Anpassung der Tests kann die entwickelte
Anwendung auch für das Ausführen und Testen von 
einigen Übungsaufgaben des Moduls Programmierung benutzt werden.
Als Erweiterung kann das Ausführen von Gradle Tasks implementiert
werden.

Das entwickelte Frontend ist ein erster Ansatz und noch sehr
ausbaufähig. Um eine vollständige Entwicklungsumgebung zu bieten,
müssen noch nützliche Funktionen implementiert werden.
Da das Backend eine REST-API bereitstellt, können
alternativ auch etablierte Texteditoren benutzt werden und ein passendes
Plugin entwickelt werden. Das hätte den großen Vorteil, dass Benutzer ihren
Quellcode in einer familiären Umgebung und mit nützlichen Werkzeugen entwickeln,
wie es bei GitHub Codespaces der Fall ist.


 
