\chapter{Grundlagen}

\section{Serverless-Computing}
1. 'Serverless Computing' Begriffserklärung\\
-- Definieren von Programmcode ohne auf Infrastruktur zu achten\\
-- Cloud-Provider übernehemen Wartung und Infrastruktur (NoOps)\\
-- Erstellen 'Serverless-App' durch kleine, abgekapselte Funktionen (FaaS),\\
oder traditionelle App in Container verpackt\\
-- Kosten pro eigentlichem Verbraucht
-- keine komplexe Verwaltung (managed Services)
-- Vergleich zu traditionellen Diensten

% TODO: Zitieren von AWS Whitepaper/Blog\\
% TODO: Zitieren von Martin Fowler\\
% TODO: Zitieren von 'What Is Serverless?' by John Chapin, Mike Roberts\\

\section{Virtual-Machine}
1. Was ist Virtualisierung?
-- Abstraktion der Hardware\\
-- Emulieren eines Betriebssystems\\
-- Machine kann mehrer Betriebssysteme laufen\\
-- Isolation vom Hostsystem\\
-- Ressourcenintensiv\\

\section{Container}
1. Was sind Container?
-- Verpacken und isolieren von Software in einer produktionsreifen Laufzeitumgebung\\
-- Erleichtert das Entwickeln auf unterschiedlichen Systemen und Umgebungen\\
-- Keine vollständige Isolation\\
-- Host und Container können sich Ressourcen teilen\\
-- Relative ressourcenarm\\

\section{Function as a Service}
-- Ein Cloud Computing Service wie PaaS, IaaS, SaaS\\
-- Kleine, stateless Funktionen die einzelnd abgerechnet und skaliert werden\\

\section{Serverless Containers}
-- Laufzeitumgebung der serverlosen Anwendung wird durch Dockerimage bereitgestellt
-- Anwendung kann beliebigen aufbau haben

\section{Open-Source und self-hosted Lösungen}
-- OpenFaaS
-- IBM OpenWhisk
-- Oracle Fn Project

\section{Sichere Umgebungen}
-- VMs sind zu Ressourcen intensiv
-- Container sind nicht sicher

\subsection{Google gVisor}
-- Applikationskernel für Container in Golang\\
-- Implementiert viele des Linux kernel interfaces\\
-- Fängt SystemCalls ab und gibt sich als GuestKernel aus\\
% TODO: siehe gVisor Doku und Whitepapers\\

\subsection{AWS Firecracker}
-- Benutzt KVM (Kernel-based Virtaul Machine, ähnlich sVMs)\\
% TODO: https://assets.amazon.science/96/c6/302e527240a3b1f86c86c3e8fc3d/firecracker-lightweight-virtualization-for-serverless-applications.pdf

\section{Eigenschaften serverloser Architekturen}
-- Architekturdartstellung einer simplen Webanwendung, Frontend + Backend + Datenbank
-- Vier eigenschaften des Serverless Computing bei jedem der Services deutlich machen
-- Serverless sieht vor soviel wie möglich die Dienste der Cloud-Provider zuverwenden