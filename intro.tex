\chapter{Einleitung}
In diesem Kapitel wird die Problemstellung und das Ziel der Arbeit vorgestellt.
Außerdem wird ein Überblick über die einzelnen Kapitel gegeben.

\section{Motivation}
Während meiner Korrektor-Tätigkeit neigte ich schnell den Quellcode der Studenten sofort zu kompilieren
und das resultierende Programm auszuführen, 
weil dadurch bereits ein sehr aussagekräftiger Eindruck über die Qualität der Abgabe ersichtlich wird. 
Jedoch bringt das Ausführen von fremden und ungeprüften Programmen einige Gefahren mit sich.
Es können Sicherheitslücken der \textit{Java Runtime Environment} (JRE) \cite{Jre}
oder des eigenen Systems ausgenutzt werden, wodurch Schadprogramme eingeschleust
werden können \cite{CveJreVuln}.
Um dies zu verhindern, kann man das Programm in einer isolierten Umgebung laufen lassen.
Das manuelle Aufsetzen einer \textit{Virtual-Machine} (VM) \cite{RedHatVM} auf dem eigenen Computer
benötigt relativ viele Ressourcen und ist zugleich auch viel Aufwand nur um ein kleines Programm auszuführen.
Das Ausweichen auf traditionelle externe Dienste von \textit{Cloud-Providern} hat den Nachteil,
dass man einen stündlichen bzw. monatlichen festen Betrag zahlen muss,
auch wenn der Dienst nicht genutzt wird. Man muss im Vorhinein grob die Auslastung abschätzen,
um den Server passend einzustellen. Außerdem muss der Server regelmäßig auf
Sicherheitslücken geprüft und dementsprechend aktualisiert werden.
Effizienter wäre eine serverlose (engl. \textit{serverless}) Anwendung, dass den
Quellcode in einer sicheren Umgebung baut und ausführt.

Das Konzept des \textit{Serverless-Computing} beinhaltet das Entwickeln und Bereitstellen
von skalierbaren Anwendungen ohne komplexe Serververwaltung, bei der nur der eigentliche Verbrauch
in Zahlung gestellt wird \cite{CioGov}.
Dafür bieten Cloud-Provider Dienste an, welche sie selbst
verwalten. Entwickler brauchen sich nur um ihre Programme sowie neue Erweiterungen kümmern,
da die mühselige Arbeit der Skalierung, Provisionierung und Instandhaltung
bereits von den Cloud-Providern übernommen wird.

\section{Ziel der Arbeit}
Das Ziel dieser Arbeit ist es eine serverlose Anwendung für das sichere Ausführen und Testen
von einfachen \textit{Java}\footnote{\url{https://www.oracle.com/de/java/}} und
\textit{Gradle}\footnote{\url{https://gradle.org/}} Konsolenanwendungen zu entwickeln.

\section{Aufbau der Arbeit}
\paragraph{Kapitel 2 - Grundlagen} Zu Beginn werden die nötigen Grundlagen und Eigenschaften von Serverless erklärt.
Ein wesentlicher Bestandteil dieser Arbeit ist, dass die Programme in einer sicheren Umgebung,
getrennt voneinander ausgeführt werden. Dafür werden
\textit{Container} und Virtual-Machines 
näher erkundet. Diese zwei Konzepte sind das Fundament der Umgebungen, in denen Cloud-Provider die 
Anwendungen ihrer Kunden sicher und getrennt voneinander ausführen.
Es werden zwei unterschiedliche serverlose Dienstarten vorgestellt, die den Entwicklern zur Verfügung stehen.
Außerdem werden aktuell verfügbare serverlose Open Source Lösungen vorgestellt.
Abschließend wird der Aufbau und die Architektur einer serverlosen Platform erläutert.

\paragraph{Kapitel 3 - Entwicklung} In diesem Kapitel wird die eigentliche Architektur
und Implementierung der entwickelten Anwendung präsentiert. Davor werden kurz die benötigten Werkzeuge
vorgestellt.

\paragraph{Kapitel 4 - Probleme und Entscheidungen} Es werden kritische Entscheidungen und
Probleme erläutert, welche die Architektur und die Implementierung der Anwendung beeinflusst haben.

\paragraph{Kapitel 5 - Evaluation} Nach dem Aufbau folgt das Testen.
Es werden Performance-Probleme aufgelistet und mögliche Verbesserungen vorgeschlagen.

\paragraph{Kapitel 6 - Ähnliche Anwendungen} Danach folgt ein Vergleich mit ähnlichen Lösungen.
Dadurch werden Anwendungsgrenzen und Erweiterungen besser hervorgehoben.

\paragraph{Kapitel 7 - Fazit} Abschließend wird ein Fazit der entwickelten Anwendung und
seiner Serverless Eigenschaften gezogen.
