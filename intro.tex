\chapter{Einleitung}

\section{Motivation}
%This is the introduction; In general, each chapter should be
%contained in another file. If you want to quote something, do it like
%Creating a footnote is easy.\footnote{An example footnote.}

Bei der Korrektur von Studentenabgaben neigt man als Korrektor schnell dazu den Quellcode
sofort zu kompiliert und auszuführen, weil dadurch bereits ein sehr aussagekräftiger Eindruck
über die Qualität der Abgabe ersichtlich wird. Jedoch bringt das Ausführen von fremden und ungeprüften Programmen einige Gefahren mit sich.
Es können Sicherheitslücken der Java Runtime Environment (JRE) \cite{Jre}
oder des eigenen Systems ausgenutzt werden, wodurch Schadprogramme eingeschleust
werden können \cite{CveJreVuln}.
Um dies zu verhindern, kann man das Programm in einer isolierten Umgebung laufen lassen.
Das manuelle Aufsetzen einer Virtual Machine (VM) auf dem eigenen Computer benötigt relativ
viele Ressourcen und ist zugleich auch recht aufwendig nur um ein kleines Programm auszuführen.
Das Ausweichen auf traditionelle, externe Dienste von Public Cloud Providern hat den Nachteil,
dass man einen stündlichen bzw. monatlichen festen Betrag zahlen muss,
auch wenn der Dienst nicht genutzt wird. Außerdem muss man im Vorhinein grob die Auslastung
abschätzen, um den Server passend zu provisionieren.
Man muss den Server selbst auf Sicherheitslücken prüfen und verwalten.
Effizienter wäre es das Programm, mithilfe einer serverlosen Anwendung, in einer sicheren Umgebung auszuführen.
Das Konzept des Serverless-Computing beinhaltet das Entwickeln und Bereitstellen
von skalierbaren Anwendungen ohne komplexe Serververwaltung, bei der nur der eigentliche Verbrauch
in Zahlung gestellt wird (Vgl. \cite{CioGov}).
Dafür bieten Public Cloud Provider Dienste an, welche sie selbst verwalten
\footnote{
  Gängige serverlose Dienste zum Beispiel: {\url{https://aws.amazon.com/de/lambda/}},\\{\url{https://cloud.google.com/run?hl=de}}
}.
Korrektoren können dadurch ihre Studentenabgaben sicher ausführen und testen.
Den Studenten steht eine einheitliche Platform zur Verfügung, in der sie ihre Arbeiten überprüfen können.
Entwickler brauchen sich nur um das Programm und neue Erweiterungen kümmern, da die mühselige Arbeit
der Skalierung, Provisionierung und Instandhaltung bereits von den Cloud Providern übernommen wird.

\section{Überblick der Arbeit}
Das Ziel dieser Arbeit ist es das Konzept des Serverless-Computing anhand eines praktischen
Beispiels zuverdeutlichen. Hierfür wird eine serverlose Applikation entwickelt,
der es den Studenten und Korrektoren des Moduls Programmierung
\footnote{
  \href{
    https://www.cs.hhu.de/de/lehrstuehle-und-arbeitsgruppen/betriebssysteme-prof-dr-michael-schoettner/lehre-und-abschlussarbeiten/fruehere-lehrveranstaltungen/vorlesungen/ws-201920/programmierung.html?C=D\%253BO\%253DA
    }{HHU Programmierung 2019/2020}
} ermöglicht einfache Java und Gradle Programme, in einer abgekapselten Umgebung, auszuführen und zu testen.

Dazu werden zu Beginn die nötigen Grundlagen und Begriffe erklärt. Es werden zwei prominente und
unterschiedliche Dienstarten vorgestellt, welche Cloud Provider aktuell den Entwicklern zur Verfügung stellen.
Es werden aktuelle serverlose Dienste der Public Cloud Provider und Open-Source Lösungen vorgestellt.
Ein wichtiger Bestandteil der Arbeit ist, dass die Abgaben der Studenten in einer sicheren Umgebung,
getrennt voneinander ausgeführt werden. Dafür werden Container und Virtual Machines
näher erkundet. Denn diese zwei Konzepte sind das Fundament der Umgebungen,
in der die Cloud Provider die zahlreichen Anwendungen ihrer Kunden sicher und getrennt voneinander ausführen.

Mithilfe der zuvor erlernten Grundlagen, vergleichen wir danach einige einfache
serverlose Architekturen mit ihren traditionellen Gegenstücken,
um einen besseren Eindruck über die Eigenschaften des Serverless-Computing zubekommen.
