\chapter{Einleitung}

\section{Motivation}
%This is the introduction; In general, each chapter should be
%contained in another file. If you want to quote something, do it like
%this:~\cite{Cavin2002a}.
Bei der Korrektur von Studentenabgaben neigt man als Korrektor schnell dazu den Quellcode
sofort zu kompiliert und auszuführen, weil dadurch bereits ein sehr aussagekräftiger Eindruck
über die Qualität der Abgabe ersichtlich wird. Jedoch bringt das Ausführen von fremden und ungeprüften Programmen einige Gefahren mit sich.
Es können Sicherheitslücken der Java Virtual-Machine (JVM)
oder des eigenen Systems ausgenutzt werden, wodurch Schadprogramme eingeschleust werden können.
Um dies zu verhindern, kann man das Programm in einer isolierten Umgebung laufen lassen.
Das manuelle Aufsetzen einer Virtual Machine (VM) auf dem eigenen Computer benötigt relativ
viele Ressourcen und ist zugleich auch recht aufwendig nur um ein kleines Programm auszuführen.
Das Ausweichen auf traditionelle, externe Dienste von Public Cloud Providern hat den Nachteil,
dass man einen stündlichen bzw. monatlichen festen Betrag zahlen muss,
auch wenn der Dienst nicht genutzt wird. Außerdem muss man im Vorhinein grob die Auslastung
abschätzen, um den Server passend zu provisionieren.
Man muss sie selbst auf Sicherheitslücken prüfen und verwalten.
Eine effizientere Lösung wäre eine serverlose Anwendung.
Das Konzept des Serverless-Computing beinhaltet das Entwickeln und Bereitstellen
von skalierbaren Anwendungen ohne komplexe Serververwaltung. 
Dafür bieten Cloud Provider Dienste an, welche von ihnen selbst verwaltet werden,
sodass sich Entwickler nur auf die Implementierung ihrer Anwendungen konzentrieren müssen.
Bei vielen Services der Public Cloud Provider, gilt zusätzlich dass nur der eigentliche Verbrauch in Zahlung gestellt wird.
Korrektoren können dadurch die Studentenabgaben sicher ausführen und testen.
Gleichzeitig stehen den Studenten eine einheitliche und verwaltungslose Platform zurverfügung,
womit sie ihre Aufgaben lösen können.

TODO: Wie 'Java Virtual-Machine' schreiben (siehe Oracle)\\
TODO: Verlinken von offizieller Oracle JVM Website\\
TODO: Self-Hosted Kubernetes Aspekt einbringen
TODO: Verlinken von OpenFaaS und Fn-Project


\section{Überblick der Arbeit}
1. Ziel der Arbeit formulieren:\\
-- Erstellen einer App zur sicheren Ausführung von Studentenabgaben\\
-- Erstellen eines Frontends (Code Editors), um den Studenten die Abgabe zu erleichtern\\
-- Maybe: Einbinden in MOPS. Darauf eingehen, dass MOPS (Java/Spring) != Python/Flask.
Parser bauen der Thymeleaf nach Jinja übersetzt\\
-- Kurz auf die Gliederung der Arbeit eingehen und Kapitel **kurz** erklären\\

