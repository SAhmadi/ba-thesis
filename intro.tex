\chapter{Einleitung}
In diesem Kapitel wird die Problemstellung und das Ziel der Arbeit vorgestellt.
Außerdem wird ein Überblick über die einzelnen Kapitel gegeben.

\section{Motivation}
Bei der Korrektur von Studentenabgaben neigt man als Korrektor schnell dazu den Quellcode sofort 
zu kompiliert und das resultierende Programm auszuführen, 
weil dadurch bereits ein sehr aussagekräftiger Eindruck über die Qualität der Abgabe ersichtlich wird. 
Jedoch bringt das Ausführen von fremden und ungeprüften Programmen einige Gefahren mit sich.
Es können Sicherheitslücken der \textit{Java Runtime Environment} (JRE) \cite{Jre}
oder des eigenen Systems ausgenutzt werden, wodurch Schadprogramme eingeschleust
werden können \cite{CveJreVuln}.
Um dies zu verhindern, kann man das Programm in einer isolierten Umgebung laufen lassen.
Das manuelle Aufsetzen einer \textit{Virtual-Machine} (VM) \cite{RedHatVM} auf dem eigenen Computer
benötigt relativ viele Ressourcen und ist zugleich auch recht aufwendig nur um ein kleines Programm auszuführen.
Das Ausweichen auf traditionelle externe Dienste von \textit{Cloud-Providern} hat den Nachteil,
dass man einen stündlichen bzw. monatlichen festen Betrag zahlen muss,
auch wenn der Dienst nicht genutzt wird. Man muss im Vorhinein grob die Auslastung abschätzen,
um den Server passend einzustellen. Außerdem muss der Server regelmäßig auf
Sicherheitslücken geprüft und dementsprechend aktualisiert werden.
Effizienter wäre eine serverlose (engl. \textit{serverless}) Anwendung, dass die
Studentenabgaben in einer sicheren Umgebung ausführt.

Das Konzept des \textit{Serverless-Computing} beinhaltet das Entwickeln und Bereitstellen
von skalierbaren Anwendungen ohne komplexe Serververwaltung, bei der nur der eigentliche Verbrauch
in Zahlung gestellt wird \cite{CioGov}.
Dafür bieten Cloud-Provider Dienste an, welche sie selbst
verwalten. Korrektoren können dadurch ihre Studentenabgaben sicher ausführen und testen.
Den Studenten steht eine einheitliche Platform zur Verfügung, in der sie ihre Arbeiten überprüfen können und
Entwickler brauchen sich nur um das Programm sowie neue Erweiterungen kümmern, da die mühselige Arbeit
der Skalierung, Provisionierung und Instandhaltung bereits von den Cloud-Providern übernommen wird.

\section{Ziel der Arbeit}
Das Ziel dieser Arbeit ist es das Konzept des Serverless-Computing anhand eines praktischen
Beispiels zu verdeutlichen. Hierfür wird eine serverlose Applikation entwickelt,
der es den Studenten und Korrektoren des Moduls
Programmierung ermöglicht einfache
\textit{Java}\footnote{\url{https://www.oracle.com/de/java/}}
und \textit{Gradle}\footnote{\url{https://gradle.org/}} Programme, in einer
abgekapselten Umgebung, auszuführen und zu testen.

\section{Aufbau der Arbeit}
\paragraph{Kapitel 2 - Grundlagen} Zu Beginn werden die nötigen Grundlagen und Begriffe erklärt.
Ein wesentlicher Bestandteil dieser Arbeit ist, dass die Abgaben der Studenten in einer sicheren Umgebung,
getrennt voneinander ausgeführt werden. Dafür werden
\textit{Container} und Virtual-Machines 
näher erkundet. Diese zwei Konzepte sind das Fundament der Umgebungen, in denen Cloud-Provider die 
Anwendungen ihrer Kunden sicher und getrennt voneinander ausführen.
Es werden zwei unterschiedliche serverlose Dienstarten vorgestellt, die den Entwicklern zur Verfügung stehen.
Außerdem werden aktuell verfügbare serverlose Dienste der Cloud-Provider sowie \textit{Open-Source} 
Lösungen vorgestellt. Mithilfe der zuvor erlernten Grundbegriffe, werden Aufbau und 
Eigenschaften einer einfachen serverlosen Architektur erläutert.

\paragraph{Kapitel 3 - Anforderungen} Es werden die minimalen Anforderungen und Funktionen
der serverlosen Anwendung vorgestellt, um den Studenten und Korrektoren ein nutzbares Werkzeug
für das Testen von Java Programmen zur Verfügung zu stellen.

\paragraph{Kapitel 4 - Entwicklung} In diesem Kapitel wird die eigentliche Architektur
und Implementierung der Anwendung präsentiert. Aufgrund des Datenschutzes ist eine
selbst betrieben Lösung sehr interessant, sodass auch kurz auf eine Open-Source Alternative eingegangen wird.

\paragraph{Kapitel 5 - Testen und Optimierung} Nach dem Aufbau folgt das Testen und Verfeinern.
Da die Anwendung primär auf Studenten und Korrektoren des Moduls Programmierung gerichtet ist,
werden einige Übungsaufgaben des Moduls aus dem Wintersemester 2019 zum Testen verwendet.
Es werden Performance-Probleme aufgelistet und mögliche Verbesserungen vorgeschlagen.

\paragraph{Kapitel 6 - Vergleiche und Evaluation} Danach folgt ein Vergleich mit ähnlichen Lösungen.
Dadurch werden Anwendungsgrenzen und Erweiterungen besser hervorgehoben.

\paragraph{Kapitel 7 - Fazit und Ausblick} Abschließend wird ein Fazit gezogen und ein Ausblick auf
zukünftige Erweiterungen und Verbesserungen der Anwendung geworfen.
