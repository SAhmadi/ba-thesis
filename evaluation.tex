\chapter{Evaluation}
In diesem Kapitel wird geprüft, ob die serverlose Anwendung
für das Ausführen und Testen von einfachen Java Programmen verwendet
werden kann. Zusätzlich wird die Anwendung geprüft,
ob die Anwendung den Serverless Prinzipien folgt.

Für das Testen und die Analyse der serverlosen Anwendung werden zwei unterschiedliche
Gradle-Projekte verwendet, welche die JUnit5 Starter Gradle Vorlage verwenden \cite{JunitStarterGradle}.
Das erste Beispielprojekt ist eine einfache
Hello World Applikation. Das Programm gibt den Text \emph{„Hello World“}
in der Standardausgabe aus. Bei dem zweiten Projekt wird
die Fibonacci-Folge\footnote{\url{https://de.wikipedia.org/wiki/Fibonacci-Folge}} berechnet.
Die beigelegten Tests prüfen stichprobenartig, ob die Folge richtet berechnet wurde.

\section{Google Cloud Run Konfiguration}
Für die Analyse werden den Cloud Run Containerinstanzen zwei vCPUs sowie zwei Gibibyte Speicher
zugewiesen. Anfragen werden nach maximal 200 Sekunden automatisch abgebrochen.
Zusätzlich wird auch die maximale Anzahl an gleichzeitigen Anfragen, welche eine Containerinstanz
entgegennehmen kann, auf 80 eingestellt.

\section{Ausführen der Programme}
Bei diesem Test werden die Beispielprojekte jeweils 10 Mal an
den Endpunkt \texttt{/run/gradle} geschickt und ausgewertet.


\section{Modulaufgaben Programmierung WS 2019}
Für das Testen der Anwendung werden Übungsaufgaben des Moduls Programmierung
der HHU aus dem Wintersemester 2019 verwendet. 

\section{Optimierungen}
Lorem ipsum dolor sit amet, consetetur sadipscing elitr,
sed diam nonumy eirmod tempor invidunt ut labore et dolore magna
aliquyam erat, sed diam voluptua.
\subsection{Docker-Image}
At vero eos et accusam et justo duo dolores et ea rebum. Stet clita kasd gubergren,
no sea takimata sanctus est Lorem ipsum dolor sit amet. Lorem ipsum dolor sit amet,
consetetur sadipscing elitr, sed diam nonumy eirmod tempor invidunt ut labore et
dolore magna aliquyam erat, sed diam voluptua.
\subsection{Cold Starts}
At vero eos et accusam et justo duo dolores et ea rebum. Stet clita kasd gubergren,
no sea takimata sanctus est Lorem ipsum dolor sit amet. Lorem ipsum dolor sit amet,
consetetur sadipscing elitr, sed diam nonumy eirmod tempor invidunt ut labore et
dolore magna aliquyam erat, sed diam voluptua.
\subsection{Gradle Daemon}
At vero eos et accusam et justo duo dolores et ea rebum. Stet clita kasd gubergren,
no sea takimata sanctus est Lorem ipsum dolor sit amet. Lorem ipsum dolor sit amet,
consetetur sadipscing elitr, sed diam nonumy eirmod tempor invidunt ut labore et
dolore magna aliquyam erat, sed diam voluptua.
\subsection{vCPU und Speicher Konfiguration}
At vero eos et accusam et justo duo dolores et ea rebum. Stet clita kasd gubergren,
no sea takimata sanctus est Lorem ipsum dolor sit amet. Lorem ipsum dolor sit amet,
consetetur sadipscing elitr, sed diam nonumy eirmod tempor invidunt ut labore et
dolore magna aliquyam erat, sed diam voluptua.

\section{Skalierbarkeit}
Lorem ipsum dolor sit amet, consetetur sadipscing elitr,
sed diam nonumy eirmod tempor invidunt ut labore et dolore magna
aliquyam erat, sed diam voluptua.
At vero eos et accusam et justo duo dolores et ea rebum. Stet clita kasd gubergren,
no sea takimata sanctus est Lorem ipsum dolor sit amet. Lorem ipsum dolor sit amet,
consetetur sadipscing elitr, sed diam nonumy eirmod tempor invidunt ut labore et
dolore magna aliquyam erat, sed diam voluptua.
At vero eos et accusam et justo duo dolores et ea rebum.
Stet clita kasd gubergren, no sea takimata sanctus est Lorem ipsum dolor sit amet.
