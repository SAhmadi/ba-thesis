\chapter{Probleme und Entscheidungen}
Während der Entwicklung von Backend und Frontend sind
einige Probleme aufgetreten, die zum Umdenken einiger Funktionen
in beiden Komponenten verneigt haben. Außerdem wurden bei der Architektur
und Funktionsweise der Anwendung kritische Entscheidungen getroffen.

\section{JUnit Tests}
Für das Testen von einzelnen Javadateien oder Gradle-Projekten werden
JUnit5 Tests vorausgesetzt. JUnit Tests lassen sie sich leicht in ein Java-Projekt integrieren und 
ausführen. Für das Ausführen von Tests der Übungsaufgaben des Moduls Programmierung
hat die Entscheidung jedoch einige Nachteile. Neben einer älteren Gradle und Java Version
verwenden die Übungsaufgaben aus dem Wintersemester 2018 keine JUnit Tests, sondern es werden
mehrere Gradle Tasks\footnote{\url{https://docs.gradle.org/current/dsl/org.gradle.api.Task.html}}
erstellt. Es werden nicht die Rückgabewerte und Zustände der Funktionen
getestet, sondern der Zustand der Standardausgabe nach dem Ausführen des Programms.
Die Übungsaufgaben müssten also angepasst werden, um von der entwickelten serverlosen 
Anwendung getestet werden zu können. Jedoch wird durch JUnit die Anwendung
auch außerhalb der Universität nutzbar und leichter erweiterbar.

\section{Gradle Wrapper}
Um ein Gradle-Projekte auszuführen und
zu testen, gibt es unter anderem zwei Möglichkeiten.
Einem Gradle-Projekt kann der Gradle-Wrapper (Wrapper)
beigelegt sein. Dieser definiert welche Version von Gradle benutzt
werden soll und lädt sich Gradle für das Bauen des Projekts selbstständig herunter.
Normalerweise ist es empfohlen den Wrapper zu benutzen,
da dieser das Bauen des Projektes standardisiert und
somit weniger Fehler entstehen \cite{GradleWrapper}.
Jedoch hat sich beim ersten Ansatz der Implementierung herausgestellt, dass der
Wrapper für das serverlose Backend etwas ungünstig ist.
Bei einer Anfrage, in der eine neue Cloud Run Containerinstanz
gestartet wird, verschwendet das Herunterladen von Gradle
unnötig Zeit und Rechenressourcen, denn die verbrauchte Zeit und der Rechenaufwand
werden bei serverlosen Diensten in Zahlung gestellt \cite{ServerlessTrends}.
Zusätzlich müssen Benutzer warten bis die Anwendung mit dem eigentlichen
Ausführen und Testen des Gradle-Projekts weitermacht.
Besser ist es Gradle in die Laufzeitumgebung einzubinden.
Dadurch muss Gradle nicht immer wieder heruntergeladen werden. Dies hat jedoch
den Nachteil, dass die auszuführenden Programme der Benutzer
kompatibel mit der verwendeten Gradle Version sein müssen.

\section{Google Cloud Run Konfiguration}
Google Cloud Run ermöglicht die Konfiguration von Ressourcen der
Containerinstanzen. Es können unter anderem die Anzahl an zugewiesenen CPUs \cite{CloudRunCpuAlloc} und
Speicher \cite{CloudRunMemLimits} angepasst werden.
Den Containerinstanzen werden zwei vCPUs sowie zwei Gibibyte\footnote{\url{https://en.wikipedia.org/wiki/Gibibyte}}
an Speicher zugewiesenen, da bei einer geringeren Einstellung Gradle
nicht genug Speicher zur Verfügung steht und Anfragen nicht bearbeitet werden.
Der Nachteil ist, dass der Dienst dadurch teurer wird,
da die Konfiguration der Ressourcen in den Kosten mit einberechnet wird \cite{CloudRunPricing}.

\section{Entfernung von Endpunkte}
Während der Entwicklung des Frontends wurde schnell ersichtlich,
dass die Erstellung eines vollständigen Gradle-Projekts nicht immer benutzerfreundlich ist.
In der Projektansicht verschwenden die Dateien und Ordner viel Platz.
Außerdem können Benutzer nicht frei über ihre Ordnerstruktur entscheiden.
Deshalb sollten die zusätzlichen Endpunkte \texttt{/run/zip} und \texttt{/test/zip}
bereitgestellt werden. Sie sollten das Ausführen und Testen von Java-Projekten mit
beliebiger Ordnerstruktur ermöglichen.
Während der Endpunkt \texttt{/run/zip} sich leicht implementieren ließ, kamen
beim Kompilieren und Ausführen der JUnit Tests Probleme hervor.
Der erste Ansatz war es alle Ordner und Unterordner des Projekts zu
durchlaufen, um alle Dateien mit der Dateiendung \emph{.java} einzusammeln.
Dateien mit \emph{Test} im Namen wurden als JUnit Testdateien gewertet. Jedoch
ist dieser Ansatz auch nicht benutzerfreundlich. Während Benutzer
bei einem Gradle-Projekt keine freie Ordnerstruktur festlegen können,
können sie bei dieser Methode nicht frei über Dateinamen entscheiden.

Eine bessere Lösung wäre es durch den Inhalt der Dateien zu ermitteln,
welche von ihnen JUnit Tests enthalten. Aufgrund der Unvollständigkeit
des \texttt{/test/zip} Endpunkts wurde der implementierte
\texttt{/run/zip} Endpunkt entfernt. Dadurch müssen Benutzer im
Frontend zwar ein vollständiges Gradle-Projekt erstellen,
um ihre Anwendung zu testen,
jedoch wird die Benutzererfahrung klar und einheitlich.
Als zusätzliche Hilfe wird den Benutzern im Frontend das Grundgerüst eines Gradle-Projekts
automatisch bereitgestellt.

\section{Zustandslosigkeit}
Bereits vor Beginn der Arbeit war eine mögliche Schwierigkeit
der Anwendung bekannt. Da die Anwendung einen serverlosen Dienst
benutzt, kann mit den auszuführenden Programmen nicht
interagiert werden. Die Containerinstanzen sind zustandslos.
Anfragen werden nicht immer von der gleichen Instanz bearbeitet.
Alle Programme, die vom Benutzer eine Eingabe einfordern,
können nicht erfolgreich ausgeführt werden. Alternativ
kann der Programmzustand in eine Datenbank ausgelagert werden
und bei der nächsten Interaktion eingelesen werden.
Jedoch wurde beschlossen die Funktion nicht in dieser Arbeit zu
implementieren. Sie kann als mögliche Erweiterung entwickelt
werden.