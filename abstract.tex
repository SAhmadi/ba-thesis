% do not change this
\pdfbookmark[0]{\IfLanguageName{ngerman}{Zusammenfassung}{Abstract}}{abstract}
\begin{center} 
    \huge \IfLanguageName{ngerman}{Zusammenfassung}{Abstract}
\end{center}

Diese Arbeit beschäftigt sich mit der Entwicklung einer
serverlosen Anwendung für das Ausführen und Testen einfacher Java
Konsolenanwendungen in einer sicheren Umgebung.
Um fremden Quellcode schnell und einfach zu testen,
wird er häufig ungeachtet jeglicher Sicherheitsrisiken
auf dem eigenen System ausgeführt. Der ausgeführte Code kann
jedoch Sicherheitslücken ausnutzen und somit dem eigenen
System schaden. Um den Quellcode von Java Konsolenanwendungen
abgesichert zu kompilieren und zu starten, wurde eine Applikation entwickelt,
die den Quellcode in einer sicheren Umgebung ausführen kann. Das
Resultat wird dem Benutzer zurückgesendet. Die Applikation
baut auf den Serverless Prinzipien auf. Dabei handelt es
sich um ein neues Cloud-Computing Konzept, mit dem
skalierbare Anwendungen ohne komplexe Serververwaltung
bereitgestellt werden können.
Bevor die eigentliche Entwicklung beginnt, wird eine Einführung
in das Thema Serverless-Computing gegeben. Es werden die
wichtigsten Eigenschaften und unterschiedliche serverlose
Dienstarten, wie FaaS, vorgestellt.

Für die Entwicklung der Anwendung wurde Cloud Run
benutzt, ein serverloser Dienst von Google. Die Anwendung
stellt eine REST-API bereit. Über unterschiedliche Endpunkte können
Benutzer ihren Java Quellcode oder ihr Gradle-Projekt versenden.
Der Java Quellcode wird durch Pythons Subprocess-Modul und dem
Java Development Kit erst kompiliert und danach ausgeführt.
Bei den Gradle-Projekten wird durch den Subprocess-Modul und den
passenden Gradle-Befehlen das Projekt gebaut und ausgeführt.
Das Resultat wird dem Benutzer zurückgesendet. Werden
zusätzlich JUnit Tests mitgesendet, können diese über
weitere Endpunkte gestartet werden. Zusätzlich zur
serverlosen Anwendung wurde ein Frontend entwickelt.
Das Frontend stellt eine kleine Entwicklungsumgebung zur Verfügung,
über die der Java Quellcode geschrieben und an das Backend versendet
werden kann.

Während der Evaluation traten einige Probleme auf.
Das Ausführen von Gradle-Projekten funktioniert nicht immer.
Bei großer Nachfrage werden ungefähr 85\% der Anfragen nicht erfolgreich
beantwortet. Es wird angenommen, dass dies Aufgrund eines Fehlers
bei dem temporären Speichern des gesendeten Zip-Archivs ist.
Außerdem braucht die serverlose Anwendung im Durchschnitt zu lange
für das Ausführen und Testen. Die Anwendung muss für eine
produktionsreife Bereitstellung verbessert und erweitert werden.
In Anbetracht der vielen Probleme kommt man zu dem Schluss,
dass die Serverless Prinzipien und serverlose Dienste
für die entwickelte Anwendung nicht geeignet sind.

