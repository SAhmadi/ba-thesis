\chapter{Anforderungen}
Um das Serverless Konzept mit ihren Vor- und Nachteil
besser zu verdeutlichen, wird in dem nachfolgenden Kapitel
eine prototypische Anwendung nach den Serverless-Prinzipien
entwickelt. Die Anwendung soll den Korrektoren und Studenten
des Moduls Programmierung der Heinrich-Heine-Universität
ermöglichen Java Quelltext auszuführen und zu testen.
Es soll möglich sein einzelne Java Dateien auszuführen.
Werden zusätzliche JUnit Testdateien mitgeliefert,
können die Tests ausgeführt werden.
Da die Aufgaben des Moduls Gradle benutzen,
sollen auch Gradle-Projekte ausgeführt und getestet werden.
Für die Ausführung wird die
Applikation-Plugin\footnote{\url{https://docs.gradle.org/current/userguide/application_plugin.html}}
vorausgesetzt.
Die Anwendung kann durch eine REST-API angesprochen werden,
stellt jedoch auch einen Endpunkt zur Verfügung, der eine Webseite zurückliefert.
Diese Seite richtet sich nach dem Styleguide der
Modularen Online Platform für Studierende (MOPS)\footnote{\url{https://mops.style}}
der HHU. In ihr wird die Entwicklungsumgebung eingebunden.

\section{Zusätzliche Funktionen}
Zusätzlich zur oben beschrieben Anwendung
wird auch eine einfache Entwicklungsumgebung entwickelt.
In ihr können Studenten Java-Projekte erstellen und ausführen.
Jedoch liegt der Fokus der Arbeit auf das serverlose Backend.