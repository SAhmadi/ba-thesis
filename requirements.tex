\chapter{Anforderungen}
Um das Serverless Kozept mit ihren Vor- und Nachteil
besser zu verdeutlichen, wird in dem nachfolgeden Kapitel
eine prototypische Anwendung nach den Serverless-Prinzipien
entwickelt. Die Anwendung soll den Korrektoren und Studenten
des Moduls Programmierung der Heinrich-Heine-Universität
ermöglichen ihre Java Programme zu testen.
Es soll möglich sein eine einzige Java Datei auszuführen.
Liefert man eine zusätzliche Datei, die JUnit Tests enthält
könnnen diese Tests ausgeführt werden und das Resultat wird
zurück geliefert. Da viele Aufgaben des Moduls
Gradle benutzen, soll auch das Überreichen eines Gradle-Projekts
ausgeführt und getestet werden können. Für das Testen, wird jedoch
das Application-Plugin\footnote{\url{gradle application plguin}}
von Gradle vorrausgesetzt.

\section{Minimale Anforderungen}
-- /run/java
-- /run/gradle
-- /test/java
-- /test/gradle
-- Ausführen von einzelnen Dateien, .zip oder gezipptes Gradleprojekt