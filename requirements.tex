\chapter{Anforderungen}
Um das Serverless Konzept mit ihren Vor- und Nachteil
besser zu verdeutlichen, wird in dem nachfolgenden Kapitel
eine prototypische Anwendung nach den Serverless-Prinzipien
entwickelt. Die Anwendung soll den Korrektoren und Studenten
des Moduls Programmierung der Heinrich-Heine-Universität
ermöglichen ihre Java Programme zu testen.
Es soll möglich sein eine einzige Java Datei auszuführen.
Liefert man eine zusätzliche Datei, die JUnit Tests enthält,
können diese Tests ausgeführt werden und das Resultat wird
zurückgeliefert. Da viele Aufgaben des Moduls
Gradle benutzen, soll auch das Überreichen eines Gradle-Projekts möglich sein.
Für das Testen wird jedoch
das Applikation-Plugin\footnote{\url{https://docs.gradle.org/current/userguide/application_plugin.html}}
von Gradle vorausgesetzt.
Die Anwendung kann durch Endpunkte angesprochen werden,
stellt jedoch auch einen Endpunkt zur Verfügung, welcher sich nach
der Modularen Online Platform für Studierende (MOPS) richtet.
In ihr wird die Entwicklungsumgebung eingebunden.

\section{Zusätzliche Funktionen}
Zusätzlich zur oben beschrieben Anwendung
wird auch eine einfache Entwicklungsumgebung entwickelt.
In ihr können Studenten Dateien erstellen und ausführen.
Jedoch liegt der Fokus der Arbeit auf das obige Backend.